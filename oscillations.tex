\subsection{Neutrino Oscillations}

Neutrino oscillation is a quantum phenomenon whereby a neutrino created with a specific lepton flavor can change into another flavor as it propagates through space.
This behavior is a direct consequence of the fact that neutrinos have mass and the flavor eigenstates \cite{Barger_Marfatia_Whisnant_2012}.


The flavor states \(\nu_\alpha\) (\(\alpha = e, \mu, \tau\)) are related to the mass states \(\nu_i\) (\(i = 1, 2, 3\)) through a unitary transformation.
This transformation can be expressed as:

\begin{align}
|\nu_\alpha\rangle &= \sum_{i} U_{\alpha i} |\nu_i\rangle,
\end{align}

where \(U_{\alpha i}\) are elements of the PMNS matrix, which is a unitary matrix describing the mixing between the flavor and mass eigenstates.

When a neutrino is produced in a flavor eigenstate \cite{Cohen_Glashow_Ligeti_2009}, it propagates as a superposition of mass eigenstates\cite{PhysRevD.97.072009}.
If we denote the neutrino state produced at \(t = 0\) as \(|\nu_\alpha\rangle\), its time evolution in terms of the mass eigenstates is given by:

\begin{align}
|\nu_\alpha(t)\rangle &= \sum_{i} U_{\alpha i} e^{-i E_i t} |\nu_i\rangle,
\end{align}

where \(E_i\) is the energy of the mass eigenstate \(\nu_i\), and \(t\) is the time of propagation.

The probability of detecting a neutrino of flavor \(\beta\) after a time \(t\) is given by the squared modulus of the amplitude \cite{Barger_Marfatia_Whisnant_2012}:

\begin{align}
P(\nu_\alpha \to \nu_\beta; t) &= \left| \langle \nu_\beta | \nu_\alpha(t) \rangle \right|^2 \\
&= \left| \sum_{i} U_{\beta i}^* U_{\alpha i} e^{-i E_i t} \right|^2.
\end{align}

Assuming the mass differences between the neutrino mass eigenstates are small compared to their energies, and using the approximation \(E_i \approx E - \frac{m_i^2}{2E}\), we can simplify the expression for the oscillation probability.
The oscillation probability for a two-flavor scenario (\(\alpha\) and \(\beta\)) is given by:

\begin{align}
P(\nu_\alpha \to \nu_\beta; L) &= \sin^2(2\theta) \sin^2\left(\frac{\Delta m^2 L}{4 E}\right),
\end{align}

where \(\theta\) is the mixing angle between the two flavors, \(\Delta m^2 = m_2^2 - m_1^2\) is the mass-squared difference, \(L\) is the distance traveled by the neutrino, and \(E\) is the neutrino's energy.
