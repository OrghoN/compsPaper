\subsection{neutrino}

Okay!

Finally!

The stage has been set!
Time for our stars --the neutrinos-- to make an appearence.

The story begins in the 1930s with Wolfgang Pauli, who first proposed the existence of neutrinos to solve a pressing problem in the field of beta decay.
He addressed the puzzle of the missing energy in beta decay experiments.
Beta decay is a process where a neutron decays into a proton, an electron, and an electron antineutrino:

\[
  n \rightarrow p + e^- + \bar{\nu}_e
\]

where \( n \) is the neutron, \( p \) is the proton, \( e^- \) is the electron, and \( \bar{\nu}_e \) is the electron antineutrino.
The problem was that the energy of the emitted beta particle (electron) and the proton did not add up to the total energy of the decaying neutron, leading to what seemed like a violation of energy conservation.

To deal with this, Pauli proposed the existence of a new, neutral particle that carried away the missing energy.
\footnote{He called this hypothetical particle the "neutron" (not to be confused with the neutron in the nucleus), but it was later renamed the "neutrino" by Enrico Fermi in 1934.}
Fermi incorporated the neutrino into his theory of beta decay, which became known as Fermi's theory of beta decay.
His theory elegantly explained the conservation of energy and angular momentum in beta decay processes.

The neutrino, denoted by \( \nu \), is a nearly massless and electrically neutral particle.
The interaction of neutrinos is governed by the weak force

For many years, neutrinos were a theoretical construct until they were finally observed experimentally by Clyde Cowan and Frederick Reines in 1956.
Their detection was achieved by capturing neutrinos emitted from a nuclear reactor and observing their interactions with a detector filled with water and cadmium chloride.

The weak interaction is described by the exchange of W and Z bosons, which mediate processes like beta decay.
The Lagrangian for the weak interaction, including the neutrino, can be written as:

\begin{align}
  \mathcal{L}_{\text{weak}} &= -\frac{g}{\sqrt{2}} \left[ \bar{\nu}_e \gamma^\mu (1 - \gamma^5) e W_\mu^+ + \bar{e} \gamma^\mu (1 - \gamma^5) \nu_e W_\mu^- \right] \nonumber \\
                            &\quad - \frac{g}{\cos \theta_W} \left[ \bar{\nu}_e \gamma^\mu (1 - \gamma^5) \nu_e Z_\mu \right]
\end{align}

where \( g \) is the coupling constant for the weak interaction, \( \theta_W \) is the Weinberg angle, and \( W_\mu^\pm \) and \( Z_\mu \) are the W and Z bosons respectively.

The 1960s introduced a new chapter with the discovery of the muon neutrino, $\nu_\mu$.
The experiment conducted by the Brookhaven National Laboratory used a beam of pions, which decay into muons and muon neutrinos:

\begin{align}
  \pi^+ \rightarrow \mu^+ + \nu_\mu
\end{align}

Here, $\pi^+$ is the positively charged pion, $\mu^+$ is the muon, and $\nu_\mu$ is the muon neutrino.
In 1962, the collaboration led by Martin LPerl and his team at the Stanford Linear Accelerator Center (SLAC) confirmed the existence of the muon neutrino by observing interactions consistent with $\nu_\mu$.

The next breakthrough in neutrino physics came in 1975 with the discovery of the tau neutrino, $\nu_\tau$.
The relevant interaction can be expressed as:

\begin{align}
  \tau^+ \rightarrow \nu_\tau + \ell^+
\end{align}

where $\tau^+$ is the positively charged tau particle, $\nu_\tau$ is the tau neutrino, and $\ell^+$ represents a lepton like a positron.
The detection of the tau neutrino was more challenging due to its lower production rates and the complexity of distinguishing it from other neutrinos.
These discoveries not only confirmed the existence of the muon and tau neutrinos but also led to the realization of the three-flavor neutrino model in the Standard Model.

