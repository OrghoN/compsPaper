\subsection{The Electron}

For the longest time, humans had thought that atoms were the smallest particle that makes up everything in the world and cannot be subdivided further\cite{Dalton}, but this idea had started to come under scrutiny by the late 1800's\cite{Pullman}.
Even then, it was thought that if anything were to make up atoms, they wouldn't be lighter than the lightest atom \cite{Thorpe}.
However, in 1897, Thomson would come in with evidence that there not only were particles that made up the atoms, but that they were on the scale of 1000 times lighter than hydrogen\cite{Thomson_1907}.
He decided to shoot cathode rays at a thermal junction so he could measure the generated heat and measured how much they deflected magnetically.
He also measured the electrical deflections by lowering the pressure in the chamber where he was measuring the deflection\cite{Thomson}.
Through these experiments, he discovered the electron
\footnote{Although Thomson did decide to call them corpuscles; a name which definitely did not stick around}
and believed that it was a fundamental part of all atoms that was very light and held a decidedly negative charge\cite{electronDiscovery}.

