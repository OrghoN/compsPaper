\subsection{Types of Detectors}

Just as there are many ways to skin a cat, neutrinos can be detected using a number of detector technologies.
Each method harnesses different physical principles and technological advancements to observe these elusive particles.

Cherenkov detectors exploit the phenomenon of Cherenkov radiation, which occurs when a neutrino interacts with a medium at speeds greater than the speed of light in that medium.
This results in the emission of a faint blue light, which can be detected and analyzed.

The Super-Kamiokande detector in Japan uses a large tank filled with ultra-pure water.
It contains thousands of photomultiplier tubes that detect the Cherenkov radiation produced when neutrinos interact with the water.

Water Cherenkov detectors are a type of Cherenkov detector specifically utilizing water as the detection medium.
These detectors are characterized by their large volumes of water and arrays of photomultiplier tubes (PMTs) arranged around the tank.

The IceCube Neutrino Observatory at the South Pole uses a cubic-kilometer array of detectors embedded in the Antarctic ice, capturing Cherenkov radiation from high-energy neutrinos.

Scintillation detectors use materials that emit light when excited by the passage of a high-energy particle.
Neutrinos interact with a scintillator material, causing it to emit flashes of light, which are then detected by photodetectors.

The NOVA detector utilizes a liquid scintillator to detect neutrinos over long baselines, aiding in the study of neutrino oscillations.

Radio detectors capture the radio waves emitted by neutrino interactions in ice or other materials.
This method is particularly useful for very high-energy neutrinos.

The ANITA (Antarctic Impulse Transient Antenna) experiment detects high-energy neutrinos via the radio waves produced by neutrino interactions with the Antarctic ice.

Liquid Argon Time Projection Chambers (LArTPCs) use liquid argon as both the detector material and the medium for drift electrons generated by neutrino interactions.
The drifted electrons are then collected and analyzed to reconstruct the interaction \cite{vonFeilitzsch2012}.

The spatial resolution in LArTPCs depends on the drift length \( d \), drift field \( E \), and the electron mobility \( \mu \).

The DUNE (Deep Underground Neutrino Experiment) will use LArTPCs to study neutrino properties with high precision, particularly in the context of long-baseline neutrino oscillation experiments.

