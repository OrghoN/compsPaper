\subsection{Dirac Vs MAJORANA Neutrinos}

To understand the nature of neutrino masses, it is crucial to delve into two different types of neutrinos; Dirac and Majorana.
So a new model was required to explain these discrepancies, and in comes Dirac.
Dirac, in 1928, extended his theory to include neutrinos, proposing that they were Dirac fermions.
According to Dirac's theory, neutrinos have distinct antiparticles, and their masses arise from the Higgs mechanism, similar to other fermions.
In the Dirac framework, the Lagrangian for neutrinos can be written as:

\begin{align}
  \mathcal{L}_{\text{Dirac}} &= \bar{\nu}_L (i\gamma^\mu \partial_\mu - m_\nu) \nu_L ,
\end{align}

where \(\nu_L\) denotes the left-handed neutrino field, \(\bar{\nu}_L\) its Dirac adjoint, and \(m_\nu\) is the Dirac mass term.
\(\nu_L\) and its antiparticle \(\bar{\nu}_L\) are distinct entities.

Then comes Majorana, who proposed in 1937 a different theory to account for neutrino masses.
Majorana suggested that neutrinos could be their own antiparticles, which leads to the Majorana condition.
In this framework, neutrinos are described by Majorana fermions.
The Majorana Lagrangian is given by:

\begin{align}
  \mathcal{L}_{\text{Majorana}} &= \frac{1}{2} \bar{\nu}_L^c (i\gamma^\mu \partial_\mu - m_\nu) \nu_L ,
\end{align}

where \(\nu_L^c\) is the charge-conjugated field of \(\nu_L\).
The Majorana mass term explicitly breaks the lepton number conservation, which is a key difference from the Dirac case.
Lepton number conservation is the principle that the total number of leptons minus antileptons remains constant in a physical process.
Majorana neutrinos are their own antiparticles, and their mass term is of the form:

\begin{align}
  \mathcal{L}_{\text{Majorana mass}} &= -\frac{1}{2} m_\nu (\nu_L^T C \nu_L ),
\end{align}

where \(C\) is the charge-conjugation matrix.


The actual nature of neutrino masses—whether Dirac or Majorana—has significant implications for our understanding of fundamental particles and the symmetries of the universe.
