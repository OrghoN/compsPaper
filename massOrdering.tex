\subsection{Neutrino Mass Ordering}

In the early days of neutrino physics, the Standard Model treated neutrinos as massless particles but experimentally we know that they have some amount of mass.
Just a very tiny amount.
Not only do we not really know the masses of the neutrinos, the ordering of their masses is still an open question.

For the mass orderingof neutrinos, we need to delve into the concept of neutrino mass eigenstates and flavor eigenstates.
Neutrinos are produced and detected in flavor eigenstates (denoted by $\nu_e$, $\nu_\mu$, and $\nu_\tau$), but they propagate as mass eigenstates ($\nu_1$, $\nu_2$, and $\nu_3$).
The relationship between these states is governed by the PMNS (Pontecorvo-Maki-Nakagawa-Sakata) matrix, which can be written as:

\begin{align}
  \begin{pmatrix}
    \nu_e \\
    \nu_\mu \\
    \nu_\tau
  \end{pmatrix}
  =
  \begin{pmatrix}
    U_{e1} & U_{e2} & U_{e3} \\
    U_{\mu1} & U_{\mu2} & U_{\mu3} \\
    U_{\tau1} & U_{\tau2} & U_{\tau3}
  \end{pmatrix}
  \begin{pmatrix}
    \nu_1 \\
    \nu_2 \\
    \nu_3
  \end{pmatrix}
\end{align}

The mass eigenstates $\nu_1$, $\nu_2$, and $\nu_3$ have different masses, but their exact ordering is not yet definitively known.
There are two possible orderings for these masses:

**Normal Ordering (NO)**: In this scenario, the masses of the neutrinos are ordered as \( m_1 < m_2 < m_3 \).
This implies that the third eigenstate, $\nu_3$, has the highest mass.
The mass differences between these states are described by:

\begin{align}
  \Delta m_{21}^2 &= m_2^2 - m_1^2 \\
  \Delta m_{32}^2 &= m_3^2 - m_2^2 \\
  \Delta m_{31}^2 &= m_3^2 - m_1^2
\end{align}

**Inverted Ordering (IO)**: Here, the masses are ordered as \( m_3 < m_1 < m_2 \).
In this case, the lightest eigenstate is $\nu_3$.
The corresponding mass differences are:

\begin{align}
  \Delta m_{21}^2 &= m_2^2 - m_1^2 \\
  \Delta m_{32}^2 &= m_2^2 - m_3^2 \\
  \Delta m_{31}^2 &= m_1^2 - m_3^2
\end{align}

Determining the correct mass ordering is essential for understanding the properties of neutrinos and has profound implications for cosmology and particle physics.
