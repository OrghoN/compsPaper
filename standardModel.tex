\subsection{The Standard Model}

Before the protagonist
\footnote{Really the ensemble cast, given that they come in three flavors, electron($e\nu$), muon ($\mu \nu$) and tau($\tau \nu$)and their respective antiparticles,}
of our story - the neutrino - can be formally introduced, the stage has to be set.
A good candidate to set the stage would be the standard model which describesthree of the four known fundamental forces, electromagnetic, weak and strong interactions (it struggles to deal with  gravity) and classifying all known elementary particles.
Just like any foundational theory that undergirds a subfield of a subject, the standard model definitely wasn't developed in a day and as such, it may behoove us to at least go over the high points ofits development in order to have a better understanding of the context that surrounds neutrinos.

One may definitely quibble about where our understanding of the fundamental particles starts from, after all, humans have been trying to find out the nature of our universe and the things that make it up going back as far as the 4th century BCE with Plato positing that everything is made up of 4 elements (water, wind, earth and fire)\cite{Timaeus} but I think it makes sense to start at the discovery of the first of the particles that made it's way into the pantheon of the standard model; the electron.
For the longest time, humans had thought that atoms were the smallest particle that makes up everything in the world and cannot be subdivided further\cite{Dalton} but this idea had started to come under scrutiny by the late 1800's.
Even then, it was thought that if anything were to make up atoms, they would n't be lighter than the lightest atom.
However, in 1897, Thomson would come in with evidence that there not only were particles that made up the atoms, but that they were on the scale of 1000 times lighter  than hydrogen.
He decided to shoot cathode rays at a thermal junction so he could measure the generated heat and neasured how much they deflectedmagnetically.
He also measured the electrical deflections by lowering the pressure in the chamber where he was measuring the deflection.
Through these experiments, discovered the electron
\footnote{Although Thomson did decide to call them corpuscles; a name which definitely did not stick around}
and believed that it was a fundamental part of all atoms that was very light and held a decidedly negative charge.\cite{electronDiscovery}