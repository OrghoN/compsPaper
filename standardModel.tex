\subsection{The Standard Model}

Before the protagonist
\footnote{Really the ensemble cast, given that they come in three flavors, electron($e\nu$), muon ($\mu \nu$) and tau($\tau \nu$)and their respective antiparticles,}
of our story - the neutrino - can be formally introduced, the stage has to be set.
A good candidate to set the stage would be the standard model which describes three of the four known fundamental forces, electromagnetic, weak and strong interactions (it struggles to deal with  gravity) and classifying all known elementary particles \cite{Oerter}.
Just like any foundational theory that undergirds a sub-field of a subject, the standard model definitely wasn't developed in a day and as such, it may behoove us to at least go over the high points of its development in order to have a better understanding of the context that surrounds neutrinos.

One may definitely quibble about where our understanding of the fundamental particles starts from, after all, humans have been trying to find out the nature of our universe and the things that make it up going back as far as the 4th century BCE with Plato positing that everything is made up of 4 elements (water, wind, earth and fire)\cite{Timaeus}, but I think it makes sense to start at the discovery of the first of the particles that made it's way into the pantheon of the standard model; the electron.
