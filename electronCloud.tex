\subsection{Electron Cloud Model}

The Schrödinger's equation would pave the way for the electron cloud model of the atom we use today, but there has to be a little more work to be done befpore we can get there.
The next step in the journey relates to Heisenberg and his famous uncertainity principle.
In his model of the atom, he  never explicitly talked about the physical position or momentum of the electron; definitely breaking with tradition. Instead, his theory focused on the observables of the electron, namely the frequency of light emitted or absorbed. He would go on to refine his uncertainity principle to state

One can never know with perfect accuracy both of those two important factors which determine the movement of one of the smallest particles—its position and its velocity (or momentum). It is impossible to accurately determine both the position and velocity of a particle at the same instant.

\begin{flushright}--- Werner Heisenberg,\end{flushright}

Or to present the idea in math form,

\begin{align}
  \Delta x \cdot \Delta p &\geq \frac{\hbar}{2} \\
  \Delta E \cdot \Delta t &\geq \frac{\hbar}{2}
\end{align}

Where $\Delta x$ stands for the uncertainty in position, $\Delta p$ stands for the uncertainty in momentum, $\Delta E$ stands for the uncertainty in energy, and $\Delta t$ stands for the uncertainty in time. $\hbar$ represents the reduced Planck constant.