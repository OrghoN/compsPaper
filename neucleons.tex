\subsection{Protons and Neutrons}

We have looked a lot at the structure of the atom and what surrounds the nucleus, but what exactly does the nucleus contain?
To do that, we have to go back in time a little bit.

In the early 20th century, it was understood that the nucleus, as proposed by Rutherford, was a dense central core of the atom.
However, this model did not fully explain the mass of the nucleus nor the nature of its internal components.
In 1917, Ernest Rutherford made a significant contribution by identifying the proton.
His experiments with alpha particles bombarding nitrogen gas led to the discovery of a new particle.
This was a positively charged particle in the center of the nucleus; What we call the proton today.

Despite Rutherford's discovery of the proton, there remained a critical question: if the nucleus contained protons, why did it not have enough charge to account for the total mass of the atom? This discrepancy led to the hypothesis of the neutron, a neutral particle within the nucleus.

In 1932, James Chadwick provided the answer by discovering the neutron.
His experiments involved bombarding beryllium with alpha particles and analyzing the resulting radiation.
He observed that this radiation was not charged and had an equivalent mass to the proton, leading to the discovery of the neutron.

Between the proton and he neutron, the charge and mass of the nucleus could be explained.
However, was that really the end of the rabbit hole or were there smaller parts that made up protons and neutrons?




