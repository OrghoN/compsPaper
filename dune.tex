\subsection{Deep Underground Neutrino Experiment (DUNE)}

DUNE is a groundbreaking experiment designed to investigate neutrino properties by utilizing an innovative approach involving a large detector placed deep underground.
The primary goal of DUNE is to study neutrino oscillations
DUNE's experimental setup involves a neutrino beam generated from a high-intensity proton accelerator at Fermilab in Illinois.
The beam travels through the Earth to a massive detector located approximately 1,300 kilometers away, deep underground at the Sanford Underground Research Facility (SURF) in South Dakota.
The large scale of the detectors allows for the precise measurement of neutrino interactions, and its underground location minimizes interference from cosmic rays, thus improving the sensitivity of the experiment \cite{DUNE_Neut_Det}.

\begin{figure}[H]
  % https://www.dunescience.org/
  \centering
  \includegraphics[width=120mm]{figures/dune.png}
  \caption{Cartoon of the DUNE setup \cite{DUNE_2020}}
  \label{dune}
\end{figure}

The plan is to have 2 sets of detectors, one at Fermilab, (near detector(ND)) and the other at SURF (far detector (FD)).
% https://arxiv.org/abs/2002.02967
The design of the DUNE far detector, grounded in cutting-edge Liquid Argon Time Projection Chamber (LArTPC) technology, is set to revolutionize particle physics.
This detector will be housed in a colossal volume of 70 kilotons of liquid argon, buried 1.5 kilometers underground \cite{DUNE_LBNF}.
To maximize the efficiency of physics experiments, the design splits this volume into four LArTPC modules, each with a usable "fiducial volume" of 10 kilotons, avoiding interactions near the edges.
To accommodate these massive detectors, approximately 800,000 tons of rock will be excavated, creating vast underground caverns.

The near detector will be built on the Argon cube concept.
The ND will have a modular design combined with a novel pixellated charge readout.
Previously, large detectors struggled with high demands for drift potentials and argon purity, which often led to risks of electric breakdown and purity losses.
By breaking down a large detector into smaller, independent modules, these risks are significantly reduced.
This modularity allows for easier maintenance and more reliable operation \cite{Biron_2020}.

\begin{figure}[H]
  % https://www.lhep.unibe.ch/research/detector_development/argoncube/index_eng.html
  \centering
  \includegraphics[width=80mm]{figures/nd.png}
  \caption{Design of the DUNE ND with $5 \times 7$ modules \cite{DUNE_2020a}}
  \label{nd}
\end{figure}

The fully pixelated charge readout adds another layer of sophistication, enabling precise event topology reconstruction.
This is important for handling high-multiplicity environments where pile-up could otherwise obscure important data.
Additionally, each module captures scintillation light to provide accurate timing information for neutrino events, further enhancing the detector's performance.

The real game-changer is the scalability of this design.
The modular approach means that the detector can be expanded to accommodate a very large active mass, opening up new possibilities for research and application.

\begin{figure}[H]
  % https://www.lhep.unibe.ch/research/detector_development/argoncube/index_eng.html
  \centering
  \includegraphics[width=80mm]{figures/ndModule.png}
  \caption{Cutaway image of a module\cite{DUNE_2020b}}
  \label{ndModule}
\end{figure}

Because of the novelty of the technology, a scaled down prototype  of the Argon cube detector called the $2 \times 2$ has been built.
Instead of having $5 \times7$ modules, it will have$2 \times 2$ modules.
Individual modules have already been built and tested before being put together to take data as part of a set.

\begin{figure}[H]
  % Original Image
  \centering
  \includegraphics[width=120mm]{figures/ndPic.png}
  \caption{Pictures of the first module to be build (Module-0)}
  \label{ndPic}
\end{figure}

The scale of DUNE and its ambitious goals are reminiscent of the dramatic shifts in scientific paradigms brought about by the discovery of subatomic particles that challenged existing theories \cite{dune_tdr}.




