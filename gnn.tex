\subsection{Graph Neural Networks}

Graph Neural Networks (GNNs) extend neural network methodologies to handle data represented in the form of graphs.
GNNs are designed to work with the complex, non-Euclidean structure of graphs.
Graphs consist of nodes (or vertices) and edges (connections between nodes), and GNNs leverage these structures to learn representations of nodes and their relationships.

To understand how GNNs function, it's important to break down their key components: node features, edge features, and the message-passing mechanism.

In a graph, each node can have associated features, which are typically represented as vectors.
If we denote the feature vector of node \( v \) as \( \mathbf{h}_v \), then the node features for all nodes in the graph can be organized into a matrix \( \mathbf{H} \), where each row corresponds to a node's feature vector.

Similarly, edges in a graph can also have features.
For an edge connecting nodes \( u \) and \( v \), we denote the edge features as \( \mathbf{e}_{uv} \).
These features can be organized into an edge feature matrix \( \mathbf{E} \).

The core idea of GNNs is the message-passing mechanism, which allows nodes to aggregate information from their neighbors.
This process involves the following steps:

\begin{enumerate}

\item Message Computation:
  For each node \( v \), we compute messages from its neighboring nodes \( \mathcal{N}(v) \).
  The message \( m_{uv} \) from node \( u \) to node \( v \) is typically computed using a function \( \phi \), which can depend on the features of both nodes and the edge between them:

  \begin{align}
    m_{uv} &= \phi(\mathbf{h}_u, \mathbf{h}_v, \mathbf{e}_{uv}) \\
           &= \phi(\mathbf{h}_u, \mathbf{e}_{uv})
  \end{align}

\item Aggregation:
  After computing the messages, each node aggregates the messages from its neighbors.
  The aggregation function \( \text{AGG} \) could be a sum, mean, or a more complex operation:

  \begin{align}
    \mathbf{m}_v &= \text{AGG}_{u \in \mathcal{N}(v)} m_{uv}
  \end{align}

\item Update:
  The aggregated message is then used to update the node’s feature vector.
  This update function \( \text{UPDATE} \) often involves a neural network layer like a fully connected layer or a more complex function:

  \begin{align}
    \mathbf{h}_v' &= \text{UPDATE}(\mathbf{h}_v, \mathbf{m}_v)
  \end{align}

  The updated feature vector \( \mathbf{h}_v' \) represents the new state of the node after considering its neighbors.

  Different GNN architectures can use various choices for the aggregation and update functions.

\end{enumerate}

Graph Neural Networks allow for the processing of graph-structured data by iteratively updating node features through message passing.
This approach allows GNNs to capture complex relationships between nodes and learn meaningful representations that are useful for various tasks such as node classification and graph classification.