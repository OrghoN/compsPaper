\subsection{Electroweak Theory}

Moving on to the other fundamental forces,  the combination of electromagnetism  and the weak force into one consistent theory  was monumental in pushing physics forward.The journey towards electroweak unification began with the discovery of the weak force, a crucial interaction responsible for processes like beta decay.
Early experiments revealed that the weak force was much weaker than electromagnetism and had a very short range.
It was eventually understood that the weak force and electromagnetism were manifestations of a more fundamental interaction.

In the 1970s, Sheldon Glashow \cite{Glashow_1959}, Abdus Salam \cite{Salam_1968}, and Steven Weinberg \cite{Weinburg_1967} formulated the electroweak theory, which successfully unified these two interactions into a single theoretical framework.
Their theory predicted the existence of the W and Z bosons, which mediate the weak force.
The electroweak theory is based on the gauge symmetry group $SU(2)_L \times U(1)_Y$.

The Lagrangian for the electroweak interaction can be written as

\begin{align}
\mathcal{L}_{\text{EW}} &= -\frac{1}{4} W^i_{\mu \nu} W^{ \mu \nu}_i - \frac{1}{4} B_{\mu \nu} B^{\mu \nu} \\
&\quad + \frac{1}{2} m_W^2 W^i_\mu W^{i \mu} + \frac{1}{2} m_Z^2 Z_\mu Z^\mu \\
&\quad - \frac{g}{2} \left( \bar{\psi}_L \gamma^\mu \frac{\tau^i}{2} W^i_\mu \psi_L \right) \\
&\quad - \frac{g'}{2} \left( \bar{\psi}_L \gamma^\mu Y \psi_L B_\mu \right) \\
&\quad - \frac{g}{2} \left( \bar{\psi}_R \gamma^\mu \frac{\tau^i}{2} W^i_\mu \psi_R \right) \\
&\quad - \frac{g'}{2} \left( \bar{\psi}_R \gamma^\mu Y \psi_R B_\mu \right) \\
&\quad + \frac{g}{2 \cos \theta_W} \left( \bar{\psi}_L \gamma^\mu \frac{\tau^i}{2} W^i_\mu \psi_L \right) \\
&\quad + \frac{g}{2 \cos \theta_W} \left( \bar{\psi}_L \gamma^\mu \frac{\tau^i}{2} Z_\mu \psi_L \right).
\end{align}

Where $W^i_{\mu \nu}$ represents the field strength tensor for the $SU(2)_L$ gauge bosons, $B_{\mu \nu}$ is the field strength tensor for the $U(1)_Y$ gauge boson, $m_W$ and $m_Z$ are the masses of the $W$ and $Z$ bosons respectively, $g$ is the $SU(2)_L$ gauge coupling constant, $g'$ is the $U(1)_Y$ gauge coupling constant, $\psi_L$ and $\psi_R$ denote the left- and right-handed fermion fields, $\tau^i$ are the Pauli matrices corresponding to the $SU(2)_L$ symmetry, $Y$ represents the hypercharge of the fermion fields, and $\theta_W$ is the Weinberg angle.



The electroweak theory successfully predicted the masses of the $W$ and $Z$ bosons, which were experimentally confirmed in 1983 at CERN.
The $W$ bosons ($W^+$ and $W^-$) mediate charged current interactions, while the $Z$ boson mediates neutral current interactions \cite{Fry_Haidt}.

The path to the electroweak theory was marked by significant experimental and theoretical advances.
In the 1930s, the discovery of the muon by Carl Anderson and Seth Neddermeyer introduced the idea that there were particles beyond the electron .
Muons were soon identified as heavier cousins of electrons, leading to the development of the concept of lepton family\cite{Neddermeyer_Anderson_1937}.

In the 1970s, the discovery of the tau lepton, a particle even heavier than the muon, further expanded the lepton family.
The tau lepton, discovered by Martin Perl and collaborators in 1975, was crucial in validating the electroweak theory \cite{tau_discovery}.
The existence of three generations of leptons (electron, muon, and tau) and their associated neutrinos was essential for the theory's development.

The electroweak unification also prompted the search for new particles, such as the Higgs boson, responsible for giving mass to the gauge bosons.
The discovery of the Higgs boson at the Large Hadron Collider in 2012 was a triumph for the Standard Model and confirmed the last missing piece of the electroweak theory.



