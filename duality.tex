\subsection{Duality of Matter}This new evidence went in the face of the wave nature of light which had been longstanding.
It seemed like  there were phenomena that could be explained by thinking of light as a wave and other phenomena that could be understood if we looked at light as a particle.
In 1924, Louis de Broglie entered the fray and decided to ask the question nobody had before.

But why not both?

He even went a step further and argued that the wave/particle dual nature was not just a thing for light, but rather all matter.

He further worked on Einstein's equation, ending up with

\begin{align}
  h \nu_0 = m_0 c^2
\end{align}

Where $h$ is the Planck constant, $\nu_0$ is the frequency, $m_0$ is the mass while $c$ is the speed of light in a vaccum. Leading to Einstein's famous equation

\begin{align}
  E = m c^2
\end{align}

Where $E$ is energy, $m$ is mass and $c$ is the speed of light in a vaccum.

This work led to the De Broglie's relationship between wavelength and momentum.

\begin{align}
  \lambda = \frac{h}{p}  
\end{align}

Where $\lambda$ means wavelength, $h$ is the Planck constant and $p$ is momentum.




