\subsection{Duality of Matter}This new evidence flew in the face of the wave nature of light, which at this point had been longstanding.
It seemed like there were phenomena that could be explained by thinking of light as a wave and other phenomena that could be understood if we looked at light as a particle.
In 1924, Louis de Broglie entered the fray and decided to ask the question nobody had before.
But why not both?
He further worked on Einstein's equation, ending up with

\begin{align}
  h \nu_0 = m_0 c^2
\end{align}

Where $h$ is the Planck constant, $\nu_0$ is the frequency, $m_0$ is the mass while $c$ is the speed of light in a vacuum.
De Broglie went on to argue that the wave/particle dual nature was not just a thing for light, but rather for \textit{all} matter\cite{DeBroglie_1925}. 

This eventually led to Einstein's famous equation

\begin{align}
  E = m c^2
\end{align}

Where $E$ is energy, $m$ is mass and $c$ is the speed of light in a vacuum\cite{Einstein_1905a}.

This work led to the de Broglie's relationship between wavelength and momentum.

\begin{align}
  \lambda = \frac{h}{p}  
\end{align}

Where $\lambda$ means wavelength, $h$ is the Planck constant and $p$ is momentum.
This relationship can be thought of as a particle travelling through space as a wave packet\cite{DeBroglie_1925}.
This hypothesis was later confirmed through cathode ray diffraction and Davison-Germer experiment\cite{Davisson_Germer_1927}\cite{Davisson_Germer_1927a}\cite{Davisson_Germer_1928}.

Erwin Schrödinger would go on to take the ideas developed by de Broglie and run with it.
He thought that if all matter can be thought of as a wave packet, there must be a wave equation to describe them.


\begin{align}
  i \hbar \frac{\partial \Psi(\mathbf{r}, t)}{\partial t} &= \left[ -\frac{\hbar^2}{2m} \nabla^2 + V(\mathbf{r}, t) \right] \Psi(\mathbf{r}, t) 
\end{align}

Where
\begin{itemize}
\item $\Psi(\mathbf{r}, t)$ stands for the wave function of the quantum system, which depends on both position $\mathbf{r}$ and time $t$.
\item $i$ is the imaginary unit.
\item $\hbar$ is the reduced Planck constant.
\item $\frac{\partial \Psi(\mathbf{r}, t)}{\partial t}$ denotes the partial derivative of the wave function with respect to time.
\item $\nabla^2$ is the Laplacian operator, representing the sum of second partial derivatives with respect to the spatial coordinates, indicating the kinetic energy term.
\item $V(\mathbf{r}, t)$ is the potential energy of the system, which may vary with position and time.
\end{itemize}

Thus the  Schrödinger's equation was born\cite{Schrödinger_1926}.