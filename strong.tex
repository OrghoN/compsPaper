\subsection{The Strong Force}

Having gone over what quarks are, the interesting question to ponder is why do quarks stay together to form hadrons?
Fundamentally, why can they not exist by themselves in a stable configuration?
The answer to both of these questions is the strong force.

It is a fundamental aspect of QCD, which is the theory describing the interactions of quarks and gluons.
Gluons are a massless particle that mediates the strong force and is part of the standard model.

The strong force is characterized by its incredibly short range but immense strength.
It operates effectively only at distances on the order of femtometers (1 fm = \(10^{-15}\) meters).

One of the key features of the strong force is quark confinement, which means that quarks are never found in isolation.
They are always bound within larger particles called hadrons.
This is due to the nature of the strong force, which becomes stronger as quarks move farther apart.
The potential energy associated with the strong force increases with distance, effectively confining quarks within hadrons.

The QCD Lagrangian, describes the strong force

\begin{align}
  \mathcal{L}_{\text{QCD}} &= -\frac{1}{4} F_{\mu\nu}^a F_a^{\mu\nu} + \bar{\psi}_i (i \gamma^\mu (D_\mu)_{ij} - m\delta_{ij}) \psi_j,
\end{align}

where \( F_{\mu\nu}^a \) is the field strength tensor for the gluon fields, \( \psi_j \) represents the quark fields, \( \gamma^\mu \) are the gamma matrices, and \( (D_\mu)_{ij} \) is the covariant derivative that includes the gluon interaction.
The term \( -\frac{1}{4} F_{\mu\nu}^a F_a^{\mu\nu} \) describes the dynamics of the gluon fields, and \( \bar{\psi}_i (i \gamma^\mu (D_\mu)_{ij} - m\delta_i) \psi_j \) describes the interaction between quarks and the gluon field.

Another fascinating property of the strong force is asymptotic freedom, which means that quarks interact more weakly at shorter distances.
This phenomenon is described by the running of the strong coupling constant \( \alpha_s \), which decreases as quarks come closer together:

\begin{align}
  \alpha_s(Q^2) &= \frac{g^2(Q^2)}{4 \pi},
\end{align}

where \( g(Q^2) \) is the running coupling constant that depends on the energy scale \( Q^2 \).
At high energies (short distances), \( \alpha_s \) becomes smaller, indicating weaker interactions between quarks.
Conversely, at low energies (larger distances), \( \alpha_s \) grows larger, leading to stronger interactions and quark confinement.

In practice, the strong force is responsible for the internal structure of hadrons.
The force between these quarks is mediated by gluons, which continuously exchange color charge and bind the quarks together in a stable configuration.
The interaction between quarks within hadrons can be described by the potential:

\begin{align}
  V(r) &= -\frac{4}{3} \frac{\alpha_s}{r}+\sigma r,
\end{align}

where \( \alpha_s \) is the strong coupling constant and \( r \) is the distance between quarks.
This potential is known as the Cornell potential and illustrates how the force increases as quarks move further apart.

